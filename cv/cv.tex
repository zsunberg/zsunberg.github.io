\documentclass[10pt,roman]{moderncv}

%% ModernCV themes
\moderncvstyle{banking}
\moderncvcolor{black}
\renewcommand{\familydefault}{\sfdefault}
\nopagenumbers{}


\renewcommand*{\addresssymbol}       {}
\renewcommand*{\mobilephonesymbol}   {}
\renewcommand*{\fixedphonesymbol}    {}
\renewcommand*{\faxphonesymbol}      {}
\renewcommand*{\emailsymbol}         {}
\renewcommand*{\homepagesymbol}      {}
\renewcommand*{\linkedinsocialsymbol}{}
\renewcommand*{\twittersocialsymbol} {}
\renewcommand*{\githubsocialsymbol}  {}


% \renewcommand\refname{Selected Publications}

%% Character encoding
\usepackage[utf8]{inputenc}

%% Adjust the page margins
\usepackage[scale=0.77]{geometry}
\usepackage{fancyhdr}
\usepackage{lastpage}
\usepackage{multicol}
\usepackage{enumitem}
\usepackage{comment}

% \includecomment{references}
\excludecomment{references}
\includecomment{research}

\pagestyle{fancy}
\fancyhf{}
\renewcommand{\headrulewidth}{0pt}
\rfoot{Zachary Sunberg \textbar{} Curriculum Vitae (\thepage/\pageref{LastPage})}

\addtolength{\textheight}{0.2in}

\usepackage[style=numeric, maxbibnames=20, defernumbers=true, sorting=none]{biblatex}
\addbibresource{mypubs.bib}

% Bold my name
% https://tex.stackexchange.com/questions/73136/make-specific-author-bold-using-biblatex?noredirect=1&lq=1
\renewcommand*{\mkbibnamegiven}[1]{%
  \ifitemannotation{me}
    {\textbf{#1}}
    {#1}}

\renewcommand*{\mkbibnamefamily}[1]{%
  \ifitemannotation{me}
    {\textbf{#1}}
    {#1}}


\DeclareFieldFormat{url}{\mkbibacro{URL}\addcolon\space\href{#1}{\faExternalLink}}
\AtEveryBibitem{
    \clearlist{address}
    \clearlist{publisher}
    \clearname{editor}
    \clearlist{organization}
    \clearfield{pages}  
    \clearlist{location}
    \clearfield{issn}
}
\newcommand{\mycventry}[4]{
    \begin{samepage}
    % \cventry{}{#3}{#2}{\mdseries{\textit{#1}}}{}{#4}
        \cventry{}{#3}{#2}{\mdseries{[#1]}}{}{#4}
        \vspace{2pt}
    \end{samepage}
}

\newcommand{\mycvitem}[2]{
    \textbf{#2}\hfill [#1]\break
}

\newcommand{\mailto}[1]{\href{mailto:#1}{#1}}

%% Personal data
\firstname{Zachary Nolan}
\familyname{Sunberg}
\address{Ann and H.\ J.\ Smead Aerospace Engineering Sciences}{University of Colorado Boulder} 
\mobile{720-933-7799}
\email{zachary.sunberg@colorado.edu}
\homepage{zachary.sunberg.net}

%%------------------------------------------------------------------------------
%% Content
%%------------------------------------------------------------------------------
\begin{document}

% to get the correct ordering
\nocite{
    lim2021voronoi,
    ahmad2021probabilistic,
    mern2021bayesian,
    sunberg2020improving,
    lim2020sparse,
peters2020alignment,
slade2020estimation,
    sonu2018hierarchy,
    sunberg2018pomcpow,
sunberg2017value,
slade2017simultaneous,
egorov2017pomdps,
sunberg2016trusted,
sunberg2016information,
sunberg2015real,
sunberg2014space,
sunberg2014real,
sunberg2013information,
sunberg2013belief,
sunberg2013fuzzy,
sunberg2012information,
}

\makecvtitle

\vspace{-2em}

\section{Academic Appointments}

\mycventry{January 2020 -- Present}{Assistant Professor}{University of Colorado, Boulder, CO}{Ann and H.\ J.\ Smead Aerospace Engineering Sciences Department}

\mycventry{October 2018 -- October 2019}{Postdoctoral Research Scholar}{University of California, Berkeley, CA}{Supervisor: Claire Tomlin, Hybrid Systems Laboratory}

\section{Education}

\mycventry{2018}{Doctor of Philosophy in Aeronautics and Astronautics}{Stanford University, Stanford, CA}{Advisor: Mykel Kochenderfer \textbar{} Thesis: ``Safety and Efficiency in Autonomous Vehicles through Planning with Uncertainty''}

\mycventry{2013}{Master of Science in Aerospace Engineering}{Texas A\&M University, College Station, TX}{Advisor: Jonathan Rogers \textbar{} Thesis: ``A Real Time Expert Control System for Helicopter Autorotation''}

\mycventry{2011}{Bachelor of Science in Aerospace Engineering}{Texas A\&M University, College Station, TX}{Summa cum Laude, Minor in Mathematics}

\section{Research}

\mycventry{January 2020 -- Present}{Sunberg Research Group}{University of Colorado Boulder}{
    \vspace{2pt}
    \cvline{Tree Search for Continuous-Space POMDPs}{Developing improved online algorithms for POMDPs with continuous state, action, and observation spaces~\cite{lim2021voronoi}.}
    \cvline{Hazard Mitigation for Self-Piloted Aerial Vehicles}{Developing POMDP approaches for responding to in-flight emergencies in unmanned aerial vehicles and personal mobility aircraft.}
    \cvline{Optimal COVID-19 Testing Strategies}{Exploring model predictive control and POMDP approaches for planning COVID-19 testing.}
    \vspace{6pt}
}

\mycventry{October 2018 -- October 2019}{Hybrid Systems Laboratory}{University of California, Berkeley, CA}{
    \vspace{2pt}
    \cvline{Strategy alignment in differential games}{
        Developed particle filtering techniques to deal with the presence of multiple Nash equilibria in differential games and coordinate strategies with humans that robots interact with~\cite{peters2020alignment}.
    }
    \cvline{Analysis of online algorithms for continuous POMDPs}{
        Developed the first proof of convergence to optimality for an online sampling-based POMDP algorithm in continuous observation spaces~\cite{lim2020sparse}.
    }
    % \cvline{Reachability-based safety constraints in partially observable domains}{
    %     Combining Hamilton-Jacobi reachability analysis with partial observability, focused on problems with human interaction; using POMDP planners to find optimized feedback policies that are aware of the reachability-based safety constraints.
    % }
    \vspace{6pt}
}

\mycventry{2015 -- 2018}{Stanford Intelligent Systems Laboratory (SISL)}{Stanford University, Stanford, CA}{
    \vspace{2pt}
    \cvline{POMCPOW}{Proved analytically that leading online POMDP solvers converge to suboptimal solutions for problems with continuous observation spaces and proposed a new algorithm, partially observable Monte Carlo planning with observation widening (POMCPOW) as a solution \cite{sunberg2018pomcpow}.}
    \cvline{Behavior-aware decision making in self-driving cars}{Showed that modeling the internal states of other human drivers and approximately solving the resulting POMDP can simultaneously improve both safety and efficiency. In particular, a multiple-lane-change maneuver on a highway can be accomplished in about half the time while still maintaining the same levels of safety and comfort \cite{sunberg2017value}. Other students are currently collaborating with me to .}
    \cvline{Adaptive control with belief space MCTS}{Solved adaptive control problems (i.e. problems where some dynamics parameters are unknown) by modeling them as POMDPs and using Monte Carlo tree search (MCTS) in the belief space. This approach achieved superior results in cases with large uncertainty \cite{slade2017simultaneous,slade2020estimation}.}
    \vspace{6pt}
}

\mycventry{2014 -- 2016}{Autonomous Systems Laboratory (ASL)}{Stanford University, Stanford, CA}{
    \vspace{2pt}
    \cvline{UAV collision avoidance}{Developed a method to dynamically optimize the performance a trusted collision avoidance system without sacrificing certifiability \cite{sunberg2016trusted}.}
    \vspace{6pt}
}

\mycventry{2013 -- 2014}{Hansen Experimental Physics Lab}{Stanford University, Stanford, CA}{
    \vspace{2pt}
    \cvline{Geostationary LISA}{Investigated gravitational coupling between a drag free test mass and a communications satellite carrying it for a laser interferometer gravity wave experiment.}
    \vspace{6pt}
}

\mycventry{2011 -- 2013}{Helicopters and Unmanned Systems Laboratory (HUSL)}{Texas A\&M University, College Station, TX}{
    \vspace{2pt}
    \cvline{Autonomus autorotation}{Created a control system for autonomous autorotation of manned and unmanned helicopters, and successfully flight tested it on a small RC helicopter \cite{sunberg2015real,sunberg2013fuzzy}.}
    \cvline{Distance metrics for Dempster Shafer theory}{Developed a distance metric for Dempster-Shafer theory that applies to orderable and continuous sets like those encountered in the real world \cite{sunberg2013belief}.}
    \vspace{6pt}
}

\mycventry{Summer 2011}{Air Force Research Lab Summer Faculty Program \mdseries{(Research Assistant)}}{Kirtland AFB, Albuquerque, NM}{
    \vspace{2pt}
    \cvline{Space situational awareness}{Developed an online algorithm for managing uncertainty about orbital vehicles and debris with a network of sensors \cite{sunberg2016information,sunberg2013information,sunberg2012information,sunberg2014space}.}
    \vspace{6pt}
}

\section{Research Funding}

\mycventry{2020 -- 2021}{PI: POMDP Algorithms for In-flight Learning in Emergencies}{NSF Center for Unmanned Aircraft Systems (C-UAS)}{
    Year 1 Amount: \$60,000 (Sep. 2020 -- Aug. 2021)\\
    PI Share: full amount
}

\section{Teaching}

\mycventry{2020 -- 2021}{ASEN 5519 Decision Making under Uncertainty}{University of Colorado Boulder}{
    Developed new entry-level graduate course about decision making under uncertainty.
}

\mycventry{2020 -- 2021}{ASEN 4018/4028 Senior Design Projects}{University of Colorado Boulder}{
    Advised senior design project teams, created new optimization-based approach for fairly creating teams based on student preferences.
}

\mycventry{June 2017}{Army High Performance Computing Summer Institute}{Stanford University, Stanford, CA}{
Developed and taught a 5 lecture course about decision making under uncertainty for college students.
}

\mycventry{2015-2017}{Stanford Artificial Intelligence Lab OutReach Summer (SAILORS, now AI4ALL)}{Stanford University, Stanford, CA}{
    Developed and taught a 2 week course and project for high school students that included programming robots for optical line following and using Dijkstra's algorithm to find the shortest path on a road network; only project mentor to serve all three years of the program. \url{http://ai-4-all.org/}
}
    
\mycventry{Autumn 2016}{AA-228/CS-238 Decision Making Under Uncertainty}{Stanford University, Stanford, CA}{
    Head course assistant for a class of around 200; developed problems for midterm project; gave guest lectures on the POMDPs.jl framework and autonomous driving research; project software was reused in a course at Iowa State University.
}

\section{Advising and Mentoring}

\subsection{Ph.D. Thesis Advisees}

\mycvitem{Spring 2020 -- Present}{Hyun Jae (Michael) Lim {\normalfont(Co-advised with Claire Tomlin at U.C. Berkeley)}}
\mycvitem{Fall 2020 -- Present}{Tyler Becker}
\mycvitem{Fall 2020 -- Present}{Qi Heng Ho}
\mycvitem{Fall 2020 -- Present}{Ben Kraske}

\subsection{M.S. Thesis Advisees}

\mycvitem{Fall 2020 -- Present}{Zakariya Laouar}
\mycvitem{Fall 2020 -- Present}{Himanshu Gupta}

\subsection{Independent Study Advisees}

\mycvitem{Fall 2020 -- Present}{Johnathan Tucker}
\mycvitem{Spring 2020}{Saurabh Mishra}

\subsection{Ph.D. Comprehensive Exam and Defense Committees}
\begin{multicols}{2}
    \small
    \textbf{2020 Defenses}\\
    Sangwoo Moon\\
    Charles (Luke) Burks\\
    Neha Garg (Nat. Univ. of Singapore, External Examiner)\\
    \vfill\null
    \columnbreak
    \textbf{2020 Comprehensive Exams}\\
    Ramya Kanlapuli\\
    Katherine Glasheen\\
    Andrew Mills\\
    Chandrakanth Venigalla
\end{multicols}
\subsection{M.S. Thesis Committees}
\begin{multicols}{2}
    \small
    Lasse Peters (TU Hamburg)\\
    Wyatt Raich\\
    Cody Charland\\
    Akash Ratheesh
\end{multicols}



\section{Department Service}

\mycventry{2020--present}{Graduate Program Committee}{Autonomous Systems Lead}{Served on Preliminary Exam Subcommittee}

\section{Invited Talks}

\textbf{2020}

{\small
    \cvline{Johns Hopkins University Applied Physics Lab, \textit{Laurel, MD}}{Some Recent Advances in Online POMDP Algorithms}
}

\textbf{2019}

{\small
\cvline{SRI International, \textit{Palo Alto, CA}}{Safety and Efficiency for Autonomous Vehicles through Online Learning}
\cvline{Washington State University, \textit{Pullman, WA}}{Safety and Efficiency for Autonomous Vehicles through Online Learning}
\cvline{University of Colorado, \textit{Boulder, CO}}{Safety and Efficiency for Autonomous Vehicles through Online Learning}
}

\textbf{2018}

{\small
\cvline{Renault-Nissan Research, \emph{Sunnyvale, CA}}{Safety and Efficiency in Autonomous Vehicles through POMDP Planning}
\cvline{Lyft Level 5, \emph{Palo Alto, CA}}{Safety and Efficiency in Autonomous Vehicles through Planning with Uncertainty}
\cvline{Makani, \emph{Alameda, CA}}{Algorithms for Uncertain, Non-convex Control Problems in the Real World}
\cvline{Indeed, \emph{San Francisco, CA}}{Safety and Efficiency in Autonomous Vehicles through Planning with Uncertainty}
}

\textbf{2017}

{\small
\cvline{Julia in Controls Workshop, \textit{ACC, Seattle, WA}}{POMDPs.jl}
\cvline{Open Source Software for Decision Making (OSS4DM), \textit{Stanford, CA}}{POMDPs.jl - Challenges and Lessons Learned}
}

\section{Industry Experience}

\mycventry{Summer 2014}{Google, Inc.\mdseries{, Mountain View, CA}}{Software Engineering Intern}{
Wrote software to evaluate and optimize a NASA collision avoidance program for use with Google self-piloted air vehicles.
}

\mycventry{Summer 2009}{Lockheed Martin Autonomous Systems\mdseries{, Littleton, CO}}{Intern}{
Helped in testing of autonomous SMSS all­terrain military transport vehicle navigation system; wrote rough terrain navigation program in C++ based on the A* search algorithm; wrote software in C++ for analyzing the performance of an advanced video analysis tool.
}

\section{Fellowships and Awards}

\mycvitem{2012-2016}{National Science Foundation Graduate Research Fellowship}
\mycvitem{February 2018}{Association for the Advancement of Artificial Intelligence Doctoral Consortium}
\mycvitem{2019}{IJCAI 2019 Distinguished Program Committee member}
\mycvitem{May 2017}{American Control Conference Student Travel Award}

\section{Open Source Software}

\mycventry{2015 -- present}{POMDPs.jl}{\normalfont{\url{https://github.com/JuliaPOMDP/POMDPs.jl}}}{
    Interface for defining continuous and discrete, fully and partially observable Markov decision processes along with a suite of state-of-the art solvers written in Julia and C++.
}

\section{Peer Review and Editing}

\mycvitem{2020 -- 2021}{Guest editor for the AIAA Journal of Aerospace Information Systems}

\cvline{Reviewer}{I have reviewed manuscripts for the following journals and conferences:}
\vspace{-2ex}
\begin{multicols}{2}
\small
Journal of Artificial Intelligence Research\\
AIAA Journal of Guidance, Control, and Dynamics\\
IEEE Robotics and Automation Letters\\
IEEE Transations on Cybernetics\\
IEEE Transactions on Intelligent Transportation Systems\\
IEEE Transactions on Intelligent Vehicles\\
Journal of Aerospace Information Systems\\
Autonomous Robots\\
Journal of the American Helicopter Society\\
International Symposium on Robotics Research\\
Intl. Conference on Robotics and Automation (ICRA)\\
Intl. Joint Conference on Artificial Intelligence (IJCAI)\\
AAAI Conference on Artificial Intelligence\\
American Control Conference (ACC)\\
Intelligent Transportation Systems Conference (ITSC)\\
Robotics, Science and Systems (RSS)\\
\end{multicols}

% \bibliographystyle{unsrt}
% \bibliography{mypubs} % a bibtex file containing the list of publications
% \printbibliography[title={Publications}, resetnumbers=true]
\printbibliography[title={Peer Reviewed Journal Publications},
                   type=article]
\printbibliography[title={Peer Reviewed Conference Publications},
                   type=inproceedings]
\printbibliography[title={Forthcoming Publications},
                   type=unpublished]


\begin{references}

\section{References}

The following people have agreed to write letters of recommendation on my behalf:\\
\begin{itemize}
    \item[] Mykel Kochenderfer, Assistant Professor of Aeronautics and Astronautics at Stanford University\\(\mailto{mykel@stanford.edu})

    \item[] Claire Tomlin, Professor of Electrical Engineering and Computer Science at the University of California, Berkeley (\mailto{tomlin@eecs.berkeley.edu})

    \item[] Marco Pavone, Assistant Professor of Aeronautics and Astronautics at Stanford University\\ (\mailto{pavone@stanford.edu})

    \item[] Suman Chakravorty, Associate Professor of Aerospace Engineering at Texas A\&M University\\ (\mailto{schakrav@tamu.edu})

    \item[] Olga Russakovsky, Assistant Professor of Computer Science at Princeton University, and Fei-Fei Li, Associate Professor of Computer Science at Stanford University and Chief Scientist of AI/ML at Google Cloud (Fei-Fei and Olga agreed to write a joint letter; please contact Olga at \mailto{olgarus@cs.princeton.edu})

\end{itemize}

\end{references}

\end{document}
