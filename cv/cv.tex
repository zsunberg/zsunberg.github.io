\documentclass[10pt,roman]{moderncv}

%% ModernCV themes
\moderncvstyle{banking}
\moderncvcolor{black}
\renewcommand{\familydefault}{\sfdefault}
\nopagenumbers{}


\renewcommand*{\addresssymbol}       {}
\renewcommand*{\mobilephonesymbol}   {}
\renewcommand*{\fixedphonesymbol}    {}
\renewcommand*{\faxphonesymbol}      {}
\renewcommand*{\emailsymbol}         {}
\renewcommand*{\homepagesymbol}      {}
\renewcommand*{\linkedinsocialsymbol}{}
\renewcommand*{\twittersocialsymbol} {}
\renewcommand*{\githubsocialsymbol}  {}


% \renewcommand\refname{Selected Publications}

%% Character encoding
\usepackage[utf8]{inputenc}

%% Adjust the page margins
\usepackage[scale=0.77]{geometry}
\usepackage{fancyhdr}
\usepackage{lastpage}
\usepackage{multicol}
\usepackage{enumitem}
\usepackage{comment}
\usepackage{nameref}
\usepackage{totcount}

% \includecomment{references}
\excludecomment{references}
\excludecomment{funding}
\excludecomment{declined}

\pagestyle{fancy}
\fancyhf{}
\renewcommand{\headrulewidth}{0pt}
\rfoot{Zachary Sunberg \textbar{} Curriculum Vitae (\thepage/\pageref{LastPage})}

\addtolength{\textheight}{0.2in}

\usepackage[style=numeric,maxbibnames=20,maxcitenames=20,defernumbers=true,sorting=none]{biblatex}
% \addbibresource[location=remote]{https://www.cu-adcl.org/bibliography/our-pubs.bib}
% \addbibresource[location=remote]{https://www.cu-adcl.org/bibliography/our-pubs.bib}
\addbibresource[location=remote]{https://raw.githubusercontent.com/CU-ADCL/CU-ADCL.github.io/main/bibliography/our-pubs.bib}
% \addbibresource[location=remote]{https://www.cu-adcl.org/bibliography/unsubmitted.bib}

% Bold my name
% https://tex.stackexchange.com/questions/73136/make-specific-author-bold-using-biblatex?noredirect=1&lq=1
\renewcommand*{\mkbibnamegiven}[1]{%
\ifitemannotation{zach} {\textbf{#1}} {\ifitemannotation{student} {\textbf{#1}} {#1}}}
\renewcommand*{\mkbibnamefamily}[1]{%
\ifitemannotation{zach} {\textbf{#1}} {\ifitemannotation{student} {\textbf{#1}} {#1}}}

\DeclareFieldFormat{url}{\mkbibacro{URL}\addcolon\space\href{#1}{\faLink}}
\AtEveryBibitem{
    \clearlist{address}
    \clearlist{publisher}
    \clearname{editor}
    \clearlist{organization}
    \clearfield{pages}  
    \clearlist{location}
    \clearfield{issn}
}

\newcommand{\mycite}[1]{[\ref{#1}]}

\newcommand{\mycventry}[4]{
    \begin{samepage}
    % \cventry{}{#3}{#2}{\mdseries{\textit{#1}}}{}{#4}
        \cventry{}{#3}{#2}{\mdseries{[#1]}}{}{#4}
        \vspace{2pt}
    \end{samepage}
}

\newcommand{\mycvitem}[2]{
    \textbf{#2}\hfill [#1]\break
}

\newcommand{\mailto}[1]{\href{mailto:#1}{#1}}

%% Personal data
\firstname{Zachary Nolan}
\familyname{Sunberg}
\address{Ann and H.\ J.\ Smead Aerospace Engineering Sciences}{University of Colorado Boulder} 
\mobile{720-933-7799}
\email{zachary.sunberg@colorado.edu}
\homepage{zachary.sunberg.net}

%%------------------------------------------------------------------------------
%% Content
%%------------------------------------------------------------------------------
\begin{document}

\makecvtitle

\vspace{-2em}

\section{Academic Appointments}

\mycventry{January 2020 -- Present}{Assistant Professor}{University of Colorado, Boulder, CO}{Ann and H.\ J.\ Smead Aerospace Engineering Sciences Department}

\mycventry{October 2018 -- October 2019}{Postdoctoral Research Scholar}{University of California, Berkeley, CA}{Supervisor: Claire Tomlin, Hybrid Systems Laboratory}

\section{Education}

\mycventry{2018}{Doctor of Philosophy in Aeronautics and Astronautics}{Stanford University, Stanford, CA}{Advisor: Mykel Kochenderfer \textbar{} Thesis: ``Safety and Efficiency in Autonomous Vehicles through Planning with Uncertainty''}

\mycventry{2013}{Master of Science in Aerospace Engineering}{Texas A\&M University, College Station, TX}{Advisor: Jonathan Rogers \textbar{} Thesis: ``A Real Time Expert Control System for Helicopter Autorotation''}

\mycventry{2011}{Bachelor of Science in Aerospace Engineering}{Texas A\&M University, College Station, TX}{Summa cum Laude, Minor in Mathematics}

\section{Industry Experience}

\mycventry{Summer 2014}{Google, Inc.\mdseries{, Mountain View, CA}}{Software Engineering Intern}{
Evaluated and optimized a NASA collision avoidance program for use with Google self-piloted air vehicles.
}

\mycventry{Summer 2009}{Lockheed Martin Autonomous Systems\mdseries{, Littleton, CO}}{Engineering Intern}{
Helped in testing of autonomous SMSS all­terrain military transport vehicle navigation system; wrote rough terrain navigation program in C++ based on the A* search algorithm; wrote software in C++ for analyzing the performance of an advanced video analysis tool.
}

\section{Awards and Fellowships}

\mycvitem{2024}{NSF CAREER Award}
\mycvitem{2023}{IET Cyberphysical Systems Journal Premium Paper Award}
\mycvitem{2021,2022}{AIAA Journal of Guidance, Control, and Dynamics Excellent Reviewer}
\mycvitem{2019}{IJCAI 2019 Distinguished Program Committee member}
\mycvitem{2018}{Association for the Advancement of Artificial Intelligence (AAAI) Doctoral Consortium}
% \mycvitem{2017}{American Control Conference Student Travel Award}
\mycvitem{2012-2016}{National Science Foundation Graduate Research Fellowship}


% \defbibcheck{jec}{\iffieldequalstr{annotation}{journal-equivalent}{}{\skipentry}}
% \defbibcheck{pr}{\iffieldequalstr{annotation}{journal-equivalent}{\skipentry}{}\iffieldequalstr{annotation}{abstract-review}{\skipentry}{}}
% \defbibcheck{ao}{\iffieldequalstr{annotation}{abstract-review}{}{\skipentry}}
% 
% \defbibcheck{jorjec}{
%   \ifentrytype{article}
%     {}
%     {\ifentrytype{inproceedings}
%       {\iffieldequalstr{annotation}{journal-equivalent}
%         {}
%         {\skipentry}}
%       {\skipentry}}
% }

\clearpage
\section{Publications}

\newcounter{jnum}
\renewcommand{\thejnum}{J\arabic{jnum}}
\newcommand{\journal}[1]{\refstepcounter{jnum}\item[{[\thejnum]}] \fullcite{#1}\par\label{#1}}
\newcounter{jcnum}
\renewcommand{\thejcnum}{JC\arabic{jcnum}}
\newcommand{\jc}[1]{\refstepcounter{jcnum}\item[{[\thejcnum]}] \fullcite{#1}\par\label{#1}}
\newcounter{cnum}
\renewcommand{\thecnum}{C\arabic{cnum}}
\newcommand{\conf}[1]{\refstepcounter{cnum}\item[{[\thecnum]}] \fullcite{#1}\par\label{#1}}
\newcounter{onum}
\renewcommand{\theonum}{A\arabic{onum}}
\newcommand{\opub}[1]{\refstepcounter{onum}\item[{[\theonum]}] \fullcite{#1}\par\label{#1}}

My name and the names of my students are indicated with bold font.
\vspace{1em}

\subsection{Peer Reviewed Journal Articles}

\begin{itemize}[labelwidth=1cm,leftmargin=1cm,itemsep=0.1em]
\journal{sharma2024investigation}
\journal{ho2024samplingbased}
\journal{lim2023optimality}
\journal{blonder2023navigation}
\journal{sunberg2022improving}
\journal{ahmad2021end}
\journal{slade2020estimation}
\journal{egorov2017pomdps}
\journal{sunberg2016information}
\journal{sunberg2015real}
\journal{sunberg2013information}
\journal{sunberg2013belief}
\end{itemize}

\subsection{Peer Reviewed Journal-Equivalent Conference Publications}
Journal-equivalent conference publications have rigorous peer review of the entire article, an acceptance rate of approximately 30\% or less, and are recognized to be as important as journal papers in the field.

\begin{itemize}[labelwidth=1cm,leftmargin=1cm,itemsep=0.1em]
\jc{ho2024recursivelyconstrained}
\jc{ho2024sound}
\jc{hong2024cieran}
\jc{mern2021bayesian}
\jc{lim2020sparse}
\jc{peters2020alignment}
\jc{sunberg2018pomcpow}
\end{itemize}

\subsection{Peer Reviewed Conference Publications}
Peer reviewed conference publications are selected based on peer review of the entire article, but are considered less rigorous and selective than journal-equivalent conference publications.

\begin{enumerate}[labelwidth=1cm,leftmargin=1cm,itemsep=0.1em]
\conf{laouar2024feasibility}
\conf{ray2024human}
\conf{deglurkar2023compositional}
\conf{ho2023simba}
\conf{kraske2023explanation}
\conf{ho2022automaton}
\conf{ho2022gaussian}
\conf{sunberg2021fair}
\conf{lim2021voronoi}
\conf{ahmad2021probabilistic}
\conf{sonu2018hierarchy}
\conf{sunberg2017value}
\conf{slade2017simultaneous}
\conf{sunberg2016trusted}
\conf{sunberg2014space}
\conf{sunberg2014real}
\end{enumerate}

\subsection{Conference Publications with Abstract-only Review}

\begin{itemize}[labelwidth=1cm,leftmargin=1cm,itemsep=0.1em]
\opub{tucker2022adaptive}
\opub{becker2022imperfect}
\opub{ray2022user}
\opub{sunberg2013fuzzy}
\end{itemize}


\section{Open Source Software} \label{sec:software}

\mycventry{2015 -- present}{POMDPs.jl}{\normalfont{\url{https://github.com/JuliaPOMDP/POMDPs.jl}}}{
    Interface for defining continuous and discrete, fully and partially observable Markov decision processes along with a suite of state-of-the art solvers written in Julia and C++ \mycite{egorov2017pomdps}. 33 direct package contributors, 44 dependent packages, 650 GitHub stars, 430 monthly downloads according to \url{juliahub.com}.
}
\mycventry{2020}{ProjectAssigner.jl}{\normalfont{\url{https://github.com/zsunberg/ProjectAssigner.jl}}}{
    Software for optimally assigning student teams to projects based on preferences, friend groups, and skills~\mycite{sunberg2021fair}. Used to assign senior project groups in the AES department since 2020.
}

\begin{funding}
\clearpage
\section{Research Funding}

\small{
Funding sources are classified by my role:
\begin{itemize}[nosep]
\item ``PI'' (\$839k total awards; \$788k my share): I am the sole or lead principal investigator for the entire project;
\item ``CU PI'' (\$4.5m total awards; \$1.4m my share): I am the lead investigator at CU, but the overall PI is at another institution;
\item ``Co-I'' (\$8.3m total awards; \$1.2m my share) I am a co-investigator, but there is another ranking PI at CU.
\end{itemize}
The first amount listed is the total award amount; if applicable, the second amount is the share specifically for the ADCL.
}
\vspace{1em}

\mycventry{2024 -- 2029}{PI: CAREER: Game-theoretic Online Planning in Partially Observable Domains}{National Science Foundation}{\$599,985}

\mycventry{2024 -- 2028}{CU PI: DECODE AI: Deception and Counter-Deception in Artificial Intelligence}{Office of Naval Research}{\$4,150,000 (total award) \textbar{} \$1,014,000 (my share), PI: Matthew Hale, Georgia Tech}

\mycventry{2023 -- 2024}{CU PI: Collaborative Research: Alternative Leaf Water use Strategies in Hot Environments}{National Science Foundation}{\$944,823 (total award) \textbar{} \$99,976 (my share), PI: Ben Blonder, UC Berkeley; (I did not participate in securing this grant, but was subcontracted to work on it later.)}

\mycventry{2023 -- 2024}{PI: Human Centered Autonomy for Dynamic sUAS Target Search Operations}{NSF Center for Autonomous Air Mobility and Sensing (CAAMS) Sub-award}{\$111,000 (total award) \textbar{} \$60,000 (my share, approx.)}

\mycventry{2023 -- 2028}{Co-I: Multi-Phenomenological, Autonomous \ldots Decision Support}{Air Force Office of Scientific Research}{\$5,000,000 (total award, approx.) \textbar{} \$500,000 (my share, approx.), PI: Marcus Holtzinger, CU Boulder; 
% Co-Is: K. Terry Alfriend (TAMU), Scott Palo, Kyle DeMars (TAMU), John Junkins (TAMU), Karen Feigh (Georgia Tech) \textbar{} 
Full title: Multi-Phenomenological, Autonomous, and Understandable SDA and XDA Decision Support}

\mycventry{2022 -- 2027}{Co-I: IUCRC Phase I: Center for Autonomous Air Mobility and Sensing (CAAMS)}{National Science Foundation}{\$2,210,225 (total award) \textbar{} \$0 (my direct share - all funds shared for travel and administrative costs), PI: Eric Frew, CU Boulder;
% Co-Is: Nisar Ahmed, Morteza Lahijanian, Sriram Sankaranarayanan
This grant establishes the IUCRC; sub-awards funded through industry contributions are listed separately.}

\mycventry{2022 -- 2025}{Co-I: Dispersed Autonomy for Marsupial Aerial Robot Teams}{National Science Foundation \textbar{} National Robotics Institute (Collaborative Research)}{\$1,045,429 (total award) \textbar{} \$300,000 (my share, approx.), PI: Eric Frew}
% \textbar{} Co-Is: Brian Argrow, Adam Houston (University of Nebraska, Lincoln)}

\mycventry{2022-2023}{PI: HIPPO (Human-Informed Planning with Probabilistic Observations)}{NSF Center for Autonomous Air Mobility and Sensing (CAAMS) Sub-award}{\$68,000}

\mycventry{2022-2024}{CU PI: SURP: Fast planning under uncertainty with operational and safety guarantees}{NASA Jet Propulsion Laboratory}{\$120,000 (total award) \textbar{} \$100,000 (my share), PI: Federico Rossi, JPL}

\mycventry{2021 -- 2023}{CU PI: Elektra: Naval Defensive Resource Allocation through POMDP Optimization}{Office of Naval Research (Subcontract of Johns Hopkins University Applied Physics Lab)}{\$169,805 \textbar{} Subcontract on larger award of unknown total amount}

\mycventry{2021-2023}{Co-I: L3Harris Modern Analytics for Mission Applications (LMA2)}{L3-Harris Technologies}{\$1,495,048 (total award) \textbar{} \$350,000 (my share, approx.), PI: Marcus Holzinger, CU Boulder (I did not participate in securing this grant, but began work on it later.)}

\mycventry{2021 -- 2022}{CU PI: Full Stack Planning and Control under Uncertainty}{NSF Center for Unmanned Aircraft Systems (C-UAS) Sub-award}{\$130,000 (total award) \textbar{} \$65,000 (my share), Co-PI Ella Atkins, Univ. of Michigan}

\mycventry{2020 -- 2021}{PI: POMDP Algorithms for In-flight Learning in Emergencies}{NSF Center for Unmanned Aircraft Systems (C-UAS) Sub-award}{\$60,000}

\end{funding}

\begin{declined}

\subsection{Declined}

\mycventry{2022}{In the moment (ITM): Aligned Algorithmic Aide (ALAI)}{Defence Advanced Research Projects Agency}{\$6,200,000 (approx., PI: Brett Israelson, Raytheon) \textbar{} My share: \$450,000 (approx.) \textbar{} CU subcontract Co-I: Nisar Ahmed}

\mycventry{2021}{PI: CAREER: Active competence through online Bayesian reasoning}{National Science Foundation}{\$599,341 \textbar{} PI share: full amount \textbar{} Full title: CAREER: Towards reliable autonomy in open environments: active competence through online Bayesian reasoning}

\mycventry{2021}{Co-I: RACER-Sim}{Defence Advanced Research Projects Agency}{Budget details are controlled unclassified information (CUI) \textbar{} PI: Christoffer Heckman; Co-I: Zachary Manchester (CMU)}

\mycventry{2020}{PI: Hazard mitigation for self-piloted vehicles...}{National Aeronautics and Space Administration}{\$806,411 \textbar{} PI share: \$400,000 (approx.) \textbar{} Full title: Hazard mitigation for self-piloted vehicles through POMDP planning and formal controller synthesis \textbar{} Co-I: Majid Zamani}

\mycventry{2020}{PI: Enabling rapid and flexible medical planning research with POMDPs.jl}{Chan Zuckerberg Initiative}{\$91,053 \textbar{} PI share: full amount}

\mycventry{2020}{Co-I: AVATAR-ACE (AViator-Agent Trusted AI for Real-time Air Combat Engagements)}{Defence Advanced Research Projects Agency}{\$5,000,000 (approx., PI: Krishna Kalyanam, Palo Alto Research Center) \textbar{} My share: \$660,000 (approx.)}

\end{declined}

% \section{Research Positions}
% 
% \mycventry{January 2020 -- Present}{Autonomous Decision and Control Laboratory (ADCL)}{Director}{University of Colorado, Boulder, CO}
% 
% \mycventry{October 2018 -- October 2019}{Hybrid Systems Laboratory}{Postdoctoral Research Scholar}{University of California, Berkeley, CA}
% 
% \mycventry{2015 -- 2018}{Stanford Intelligent Systems Laboratory (SISL)}{Graduate Research Assistant}{Stanford University, Stanford, CA}
% 
% \mycventry{2014 -- 2016}{Autonomous Systems Laboratory (ASL)}{Graduate Research Assistant}{Stanford University, Stanford, CA}
% 
% \mycventry{2013 -- 2014}{Hansen Experimental Physics Lab}{Graduate Research Assistant}{Stanford University, Stanford, CA}
% 
% \mycventry{2011 -- 2013}{Helicopters and Unmanned Systems Laboratory (HUSL)}{Graduate Research Assistant}{Texas A\&M University, College Station, TX}
% 
% \mycventry{Summer 2011}{Air Force Research Lab Summer Faculty Program}{Research Assistant}{Kirtland AFB, Albuquerque, NM}

\section{Invited Talks and Panels}

\newcounter{inum}
\renewcommand{\theinum}{I\arabic{inum}}
\newcommand{\invited}[3]{\refstepcounter{inum}\item[{[\theinum]}] \textbf{#1:} ``#2''. #3}

% \newcounter{invited}
% \regtotcounter{invited}
% \newcounter{conference}
% \regtotcounter{conference}
% \newcounter{panel}
% \regtotcounter{panel}
% Totals: \total{invited} invited talks, \total{panel} panel, \total{conference} other contributed talks.
% \vspace{1ex}


\begin{itemize}[labelwidth=1cm,leftmargin=1cm,itemsep=0.1em]
    \invited{Autonomy Talks Series, Massachusetts Institute of Technology (MIT) \textit{(Online)}}{Breaking the curse of dimensionality in POMDPs with sampling-based online planning} {2024. \url{https://www.youtube.com/watch?v=XDIyX0tanjk}}
    \invited{University of California, San Diego}{Breaking the curse of dimensionality in POMDPs with sampling-based online planning}{2024}
    \invited{University of Illinois Urbana-Champaign (UIUC)}{Safe and efficient autonomy in the face of uncertainty}{2022}
    \invited{University of Texas at Austin}{Safe and efficient autonomy in the face of uncertainty}{2022}
    \invited{Korea Advanced Institute of Science and Technology (KAIST) \textit{(Online)}}{Safe and efficient autonomy in the face of state and interaction uncertainty}{2022}
    \invited{AIAA Rocky Mountain Annual Technical Symposium}{Machine Learning in Aerospace Systems}{2021 (panel)}
    \invited{NASA Jet Propulsion Lab, \textit{Pasadena, CA}}{Scalable online POMDP planning for safe and efficient autonomy}{2021}
    \invited{Johns Hopkins University Applied Physics Lab, \textit{Laurel, MD}}{Some Recent Advances in Online POMDP Algorithms}{2021}
    \invited{SRI International, \textit{Palo Alto, CA}}{Safety and Efficiency for Autonomous Vehicles through Online Learning}{2019}
    \invited{Washington State University, \textit{Pullman, WA}}{Safety and Efficiency for Autonomous Vehicles through Online Learning}{2019}
    \invited{University of Colorado, \textit{Boulder, CO}}{Safety and Efficiency for Autonomous Vehicles through Online Learning}{2019}
    \invited{Renault-Nissan Research, \emph{Sunnyvale, CA}}{Safety and Efficiency in Autonomous Vehicles through POMDP Planning}{2018}
    \invited{Lyft Level 5, \emph{Palo Alto, CA}}{Safety and Efficiency in Autonomous Vehicles through Planning with Uncertainty}{2018}
    \invited{Makani, \emph{Alameda, CA}}{Algorithms for Uncertain, Non-convex Control Problems in the Real World}{2018}
    \invited{Indeed, \emph{San Francisco, CA}}{Safety and Efficiency in Autonomous Vehicles through Planning with Uncertainty}{2018}
\end{itemize}

\section{Other Presentations}
{\small
\cvline{JuliaCon 2021}{``POMDPs.jl and Interactive Assignments in Julia''. 2021}
\cvline{Julia in Controls Workshop, \textit{ACC, Seattle, WA}}{``POMDPs.jl''. 2017}
\cvline{Open Source Software for Decision Making (OSS4DM), \textit{Stanford, CA}}{``POMDPs.jl - Challenges and Lessons Learned''. 2017}
}

\section{Teaching}
\subsection{At CU Boulder}
\mycventry{Fa 2022, Sp 2024}{ASEN 3728/3128 Aircraft Dynamics}{University of Colorado Boulder, typical enrollment: 140}{
    Required Junior-level course in aircraft dynamics, stability, and control.\\
    Team taught with Prof.~Eric Frew in 2022.\\
    Open-source course materials: \url{https://github.com/zsunberg/Aircraft-Dynamics-Materials}
}

\mycventry{Sp 2020, Sp 2021, Sp 2022, Sp 2023, Sp 2024}{ASEN 5264 Decision Making under Uncertainty}{University of Colorado Boulder, typical enrollment: 30-50}{
    New entry-level graduate course about decision making under uncertainty created by me.\\
    Open-source course materials: \url{https://github.com/zsunberg/CU-DMU-Materials}\\
    Open-source companion software package: \url{https://github.com/zsunberg/DMUStudent.jl}
}

\mycventry{Fa 2021, Fa 2023}{ASEN 6519 Advanced Survey of Sequential Decision Making}{University of Colorado Boulder, typical enrollment: 10-15}{
    New advanced graduate course that surveys recent advances in decision making under uncertainty created by me. 
}

\mycventry{Fa 2020-Sp 2021}{ASEN 4018/4028 Senior Design Project}{University of Colorado Boulder, typical enrollment: 250-300, team size: 9-12}{
    Capstone senior design course. I mentored two teams and created new optimization-based approach (see Open Source Software) for fairly creating teams based on student preferences~\mycite{sunberg2021fair}.
}

\subsection{Prior to CU Boulder}

\mycventry{June 2017}{Army High Performance Computing Summer Institute}{Stanford University, Stanford, CA}{
Developed and taught a 5 lecture course about decision making under uncertainty for college students.
}

\mycventry{2015-2017}{Stanford Artificial Intelligence Lab OutReach Summer (SAILORS, now AI4ALL)}{Stanford University, Stanford, CA}{
    Developed and taught a 2 week course and project for high school students that included programming robots for optical line following and using Dijkstra's algorithm to find the shortest path on a road network; only project mentor to serve all three years of the program. \url{http://ai-4-all.org/}
}
    
\mycventry{Autumn 2016}{AA-228/CS-238 Decision Making Under Uncertainty}{Stanford University, Stanford, CA}{
    Head course assistant for a class of around 200; developed problems for midterm project; gave guest lectures on the POMDPs.jl framework and autonomous driving research; project software reused at Iowa State University.
}

\section{Advising and Mentoring}

\subsection{Postdoctoral Scholars}
\mycvitem{2024 -- present}{Ofer Dagan}

\subsection{Graduated Ph.D. Thesis Advisees}
\mycventry{2020 -- 2023}{Hyun Jae (Michael) Lim {\normalfont (Co-advised with Claire Tomlin at U.C. Berkeley)}}{AI Scientist, C3.ai}{}

\subsection{Current Ph.D. Thesis Advisees}
\mycvitem{Fall 2020 -- Present}{Qi Heng Ho {\normalfont(Comprehensive exam spring 2024)}}
\mycvitem{Fall 2020 -- Present}{Tyler Becker {\normalfont(Prelim exam fall 2021)}}
\mycvitem{Fall 2020 -- Present}{Ben Kraske {\normalfont(Prelim exam fall 2021, Received NSF GRFP)}}
\mycvitem{Fall 2021 -- Present}{Zakariya Laouar {\normalfont(Prelim exam fall 2022)}}
\mycvitem{Fall 2020 (Started as MS) -- Present}{Himanshu Gupta {\normalfont(Prelim exam fall 2023)}}
\mycvitem{Summer 2022 -- Present}{Jackson Wagner {\normalfont(Attempting prelim exam fall 2024)}}

\subsection{Graduated M.S. Thesis Advisees}
\mycventry{Spring 2022 -- Fall 2022}{William Pope}{U.S. Space Force Officer}{}

\subsection{Independent Study Advisees}
\mycvitem{Fall 2020 -- Summer 2022}{Johnathan Tucker {\normalfont(Received NSF GRFP under my direction)}}
\mycvitem{Spring 2020}{Saurabh Mishra}
\mycvitem{2023 -- 2024}{Austin Monell}

\begin{minipage}{\textwidth}
\subsection{Ph.D. Comprehensive Exam and Defense Committees}
\begin{multicols}{2}
    \small
    Sangwoo Moon\\
    Charles (Luke) Burks\\
    Neha Garg (Nat. Univ. of Singapore, External Examiner)\\
    Ramya Kanlapuli\\
    Katherine Glasheen\\
    Andrew Mills\\
    Chandrakanth Venigalla\\
    Sam Fedeler\\
    John Jackson\\
    Prashin Sharma (Univ. of Michigan)\\
    Shakeeb Ahmad\\
    John Mern (Stanford Univ.)\\
    Aastha Acharya\\
    Shohei Wakayama\\
    Marcus Lapeyrolerie (Univ. of California, Berkeley, Quals)
    John R. Martin\\
    Adam Herrmann\\
    Camron (Alex) Hirst\\
    Hunter Ray\\
\end{multicols}
\end{minipage}\\
\\

\subsection{M.S. Thesis Committees}
\begin{multicols}{2}
    \small
    Lasse Peters (TU Hamburg)\\
    Wyatt Raich\\
    Cody Charland\\
    Akash Ratheesh\\
    Jamison Mcginley\\
    Rio McMahon\\
    Eli Kravitz\\
    Abdoulaye Diallo
\end{multicols}

\section{Department Service}

\mycvitem{2023--present}{Onboarding mentor for new faculty member}
\mycvitem{2023--present}{Autonomous Systems Faculty Search Committee}
\mycvitem{2023--present}{Undergraduate Operations Committee}
\mycventry{2020--2023}{Graduate Program Committee}{Autonomous Systems Lead}{Served on Preliminary Exam Subcommittee, Revised MS Admissions Criteria}

\section{Outreach and Inclusion}

\mycvitem{2023}{Smead aerospace career panelist}
\mycvitem{2022}{Rising Stars in Aerospace organizing committee member}
\mycvitem{2021, 2022}{Speaker for Tuskegee Airmen outreach event at CU}
\mycvitem{2021}{Invited Speaker for AIAA Movie Night \& Technical Discussion: 2001: A Space Odyssey}

\section{Conference and Workshop Organization}

\mycventry{2023}{Inference and Decision Making for Autonomous Vehicles (IDMAV)}{Workshop at Robotics: Science and Systems (RSS)}{Co-organizer with Christoffer Heckman, Han-Lim Choi (KAIST), and students}

\mycventry{2022}{Strategic multi-agent interactions: game theory for robot learning and decision making}{Workshop at the Conference on Robotic Learning (CoRL)}{Co-organizer with David Fridovich-Keil (Univ. of Texas), Negar Mehr (Univ. of Illinois), and Forrest Laine (Vanderbilt)}

\section{Academic Peer Review and Editing}

\mycvitem{2020 -- 2022}{Guest editor for the AIAA Journal of Aerospace Information Systems}

\cvline{Reviewer}{I have reviewed manuscripts for the following journals and conferences:}
\vspace{-2ex}
\begin{multicols}{2}
\small
AAAI Conference on Artificial Intelligence\\
AIAA Journal of Guidance, Control, and Dynamics\\
AIAA Journal of Aerospace Information Systems\\
American Control Conference (ACC)\\
Journal of the American Helicopter Society\\
Autonomous Robots\\
Artificial Intelligence\\
Journal of Artificial Intelligence Research\\
Field Robotics\\
IEEE Robotics and Automation Letters\\
IEEE Conference on Decision and Control (CDC)\\
IEEE Transactions on Cybernetics\\
IEEE Transactions on Intelligent Transportation Systems\\
IEEE Transactions on Intelligent Vehicles\\
Intelligent Transportation Systems Conference (ITSC)\\
Intl. Journal of Robotics Research (IJRR)\\
Intl. Symposium on Robotics Research\\
Intl. Conference on Robotics and Automation (ICRA)\\
Intl. Joint Conference on Artificial Intelligence (IJCAI)\\
Learning for Decision and Control Conference (L4DC)\\
Operations Research\\
R (programming language) Journal\\
Robotics, Science and Systems (RSS)\\
\end{multicols}

\section{Proposal Review}
\mycvitem{2023}{NSF CISE Proposal Review Panelist}
\mycvitem{2023}{Army Research Office (ARO) Proposal Reviewer}

\end{document}
